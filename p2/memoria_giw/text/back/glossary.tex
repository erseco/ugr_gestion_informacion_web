\chapter{Glosario de términos}

\textbf{MySQL}: sistema de gestión de bases de datos relacional desarrollado bajo licencia dual GPL/Licencia comercial por Oracle Corporation.
\bigskip

\textbf{Hash}: también llamadas funciones de resumen son algoritmos que consiguen crear a partir de una entrada (ya sea un texto, una contraseña o un archivo, por ejemplo) una salida alfanumérica de longitud normalmente fija que representa un resumen de toda la información que se le ha dado (es decir, a partir de los datos de la entrada crea una cadena que solo puede volverse a crear con esos mismos datos).
\bigskip

\textbf{Honeypot}: Un honeypot, o sistema trampa, es una herramienta de la seguridad informática dispuesto en una red o sistema informático para ser el objetivo de un posible ataque informático, y así poder detectarlo y obtener información del mismo y del atacante.
\bigskip

\textbf{Oracle Database}: es un sistema de gestión de base de datos de tipo objeto-relacional, desarrollado por Oracle Corporation.
\bigskip

 \textbf{Objeto-Relacional}: es una extensión de la base de datos relacional tradicional, a la cual se le proporcionan características de la programación orientada a objetos.
\bigskip

\textbf{Software libre}: software cuya licencia permite que este sea usado, copiado, modificado y distribuido libremente según el tipo de licencia que adopte.
\bigskip
