\addcontentsline{toc}{chapter}{Bibliografía}

\subsubsection*{Libros consultados durante la realización del proyecto:}


%http://ruder.io/optimizing-gradient-descent/

\bibitem{stevekrug}
Steve Krug.
\newblock {\em Don't make me think! : a common sense approach to web usability.}
\newblock New Riders, 2006.
\newblock ISBN: 0321344758.

\bibitem{jakonielsen}
Jakob Nielsen.
\newblock {\em 10 Usability Heuristics for User Interface Design.}
\newblock Nielsen Norman Group, 1995.

\bibitem{melenaalva}
María Elena Alva Obeso.
\newblock {\em Metodología de medición y evaluación de la usabilidad en sitios Web educativos.}
\newblock Universidad de Oviedo, 2005.
\newblock ISBN: 9788483175842.

\bibitem{ericreiss}
Eric Reiss.
\newblock {\em Usable usability: simple steps for making stuff better .}
\newblock John Wiley \& Sons, Inc., 2012.
\newblock ISBN: 9781118185476.


\subsubsection*{Artículos sobre las metodologías utilizadas en el proyecto:}

\bibitem{art_14} `'The Distributed Course'. Stephen Downes. 2008. \url{https://sites.google.com/site/themoocguide/3-cck08---the-distributed-course}

\bibitem{art_13} ``Top Universities Test the Online Appeal of Free''. Richard Pérez-Peña. 18/07/2012. \url{http://www.nytimes.com/2012/07/18/education/top-universities-test-the-online-appeal-of-free.html}

\bibitem{art_12} `'Cuando un profesor de la Gran Depresión predijo la educación online'. Jonathan Préstamo. 09/06/2017. \url{http://www.teknoplof.com/2017/06/09/cuando-profesor-la-gran-depresion-predijo-la-educacion-online/}

\bibitem{art_11} ``Legislación sobre accesibilidad web en España, Europa y otros países''. Olga Carreras Montoto. 11/12/2016. \url{https://olgacarreras.blogspot.com.es/2005/01/referencia-sobre-legislacin-espaola.html}

\bibitem{art_11} ``Having problems with SSL by René Breedveld''. René Breedveld. 30/10/2015. \url{https://github.com/dmuneras/moodle-theme_archaius/wiki/Having-problems-with-SSL-byRen\%C3\%A9-Breedveld}

\bibitem{art_10} ``An user can enter with an admin role only copying valid sessionID from another computer with the same IP address.''. Juan Carlos Molina Giménez, 18/01/2016. \url{https://tracker.moodle.org/browse/MDL-52812}

\bibitem{art_09} ``mdl\_log now a legacy table''. Stuart Mealor, 30/05/2014. \url{http://elearningblog.moodlebites.com/mod/oublog/viewpost.php?post=90}

\bibitem{art_08} ``ARP spoofing''. Wikipedia, última edición: 01/05/2017 \url{https://en.wikipedia.org/wiki/ARP_spoofing}

\bibitem{art_07} ``ARP and ICMP redirection games''. Yuri Volobuev, 19/11/1997. \url{http://insecure.org/sploits/arp.games.html}

\bibitem{art_06} ``Presto Parking: ¿Un sitio web PCI-Compliant sin HTTPs?''. José C. A., 17/04/2017. \url{http://www.elladodelmal.com/2017/04/presto-parking-un-sitio-web-pci.html}

\bibitem{art_05} ``Corregidas cuatro vulnerabilidades en Moodle''. Antonio Ropero, 3/04/2017. \url{http://unaaldia.hispasec.com/2017/04/corregidas-cuatro-vulnerabilidades-en.html}

\bibitem{art_04} ``Múltiples vulnerabilidades en Moodle''. Antonio Ropero, 25/11/2016. \url{http://unaaldia.hispasec.com/2016/11/multiples-vulnerabilidades-en-moodle.html}

\bibitem{art_03} ``Important Announcement Regarding YUI''. Julien Lecomte, 29/08/2014. \url{https://yahooeng.tumblr.com/post/96098168666/important-announcement-regarding-yui}

\bibitem{art_02} ``Learning logs: How long are your users online? Analytics Part 2''. James Ballard, 01/06/2015. \url{https://infiniterooms.wordpress.com/2015/06/01/learning-logs/}

\bibitem{art_01} ``Reckon you've seen some stupid security things? Here, hold my beer...''. Troy Hunt, 28/04/2017. \url{https://www.troyhunt.com/reckon-youve-seen-some-stupid-security-things-here-hold-my-beer/}

\bigskip
\subsubsection*{Páginas de consulta sobre licencias, desarrollo y uso del software analizado}
\bibitem{CC} {\tt Creative Commons Share Alike 4.0}. \url{https://creativecommons.org/licenses/by-sa/4.0/}
\bibitem{moodle} {\tt moodle}. \url{https://docs.moodle.org}
\bibitem{wikibooks} Wikibooks ({\tt LaTeX}). \url{https://en.wikibooks.org/wiki/LaTeX}
\bibitem{ettercap} {\tt ettercap}. \url{https://github.com/Ettercap/ettercap}
\bibitem{statuscake} {\tt StatusCake}. \url{https://statuscake.com/kb}
\bibitem{urlscan} {\tt urlscan.io}. \url{https://urlscan.io/about/}
\bibitem{moodleplugin} {\tt material download moodle plugin}. \url{https://github.com/TUM-MZ/moodle-block_material-download}
\bibitem{moodletheme} {\tt archaius theme}. \url{https://moodle.org/plugins/theme_archaius}
\bibitem{rae} {\tt Diccionario RAE}. \url{http://dle.rae.es/}
\bibitem{w3c} {\tt W3C}. \url{https://www.w3.org/}
\bibitem{wcag} {\tt WCAG}. \url{https://www.w3.org/TR/WCAG/}
\bibitem{taw} {\tt T.A.W.}. \url{http://www.tawdis.net/}
\bibitem{tenon} {\tt tenon.io}. \url{https://tenon.io}
\bibitem{codesniffer} {\tt HTML CodeSniffer}. \url{http://squizlabs.github.io/HTML_CodeSniffer/}
\bibitem{openwrt} {\tt OpenWRT}. \url{https://wiki.openwrt.org/toh/huawei/hg556a}
\bibitem{moodledb} Moodle Database Schema. \url{https://docs.moodle.org/dev/Database_Schema}
\bibitem{mooc} MOOC. \url{https://es.wikipedia.org/wiki/Massive_Open_Online_Course}




\bigskip
\subsubsection*{Otro material}
\begin{itemize}
	\item Diversas consultas puntuales al sitio {\tt Stack OverFlow}.
	\item Material docente de las asignaturas \textbf{Fundamentos de Ingeniería del Software}, \textbf{Seguridad en Sistemas Operativos}, \textbf{Desarrollo de Aplicaciones para Internet}, \textbf{Diseño y Desarrollo de Sistemas de Información}, \textbf{Seguridad en Sistemas Operativos} y \textbf{Sistemas de Información Basados en Web} impartidas en \textbf{Grado en Ingeniería Informática} en la \textbf{Universidad de Granada}.
\end{itemize}