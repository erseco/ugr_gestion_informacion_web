\chapter{Introducción}

\subsubsection{En esta práctica se construirá un sistema de recuperación
de información empleando la biblioteca Lucene, compuesto de dos
programas:}

\begin{enumerate}
\item{
  Un \textbf{indexador}, el cual recibirá como argumentos la ruta de la
  colección documental a indexar, el fichero de palabras vacías a
  emplear y la ruta donde alojar los índices, y llevará a cabo la
  indexación, creando los índices oportunos y ficheros auxiliares
  necesarios para la recuperación. Esta aplicación se ejecutará en la
  línea de mandatos y no tendrá ningún componente gráfico. Este software
  realizará las tareas de tokenización, eliminación de palabras vacías y
  extracción de raíces antes de crear el índice.}
\item
  Un \textbf{motor de búsqueda}, que al ejecutarse recibirá como
  argumento la ruta donde está alojado el índice de la colección y
  permitirá que un usuario realice una consulta de texto y obtenga el
  conjunto de documentos relevantes a dicha consulta. En este caso, el
  programa sí será gráfico. Sobre la consulta se realizarán los mismos
  procesos que sobre los documentos en el indexador.
\end{enumerate}