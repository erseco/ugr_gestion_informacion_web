\begin{center}
{\LARGE\bfseries\titulo}\\
\end{center}
\begin{center}
\autor\
\end{center}

\section*{Resumen}

\bigskip
\noindent{\textbf{Palabras clave}: \textit{\keywords}\\

\subsection*{Los objetivos de esta práctica son:}

\begin{enumerate}
\item
  Conocer las partes principales que tiene un sistema de recuperación de
  información y qué funcionalidad tiene cada una.
\item
  Implementar un sistema de recuperación de información.
\item
  Emplear la biblioteca \texttt{Lucene} para facilitar dicha
  implementación.
\end{enumerate}


\newpage
\thispagestyle{empty}
\
\vspace{3cm}

\noindent\rule[-1ex]{\textwidth}{2pt}\\[4.5ex]

Yo, \textbf{\autor}, alumno de la titulación \textbf{\grado} de la \textbf{\escuela\ de la \universidad}, autorizo la ubicación de la siguiente copia de mi Trabajo (\textit{\titulo}) en la biblioteca del centro para que pueda ser consultada por las personas que lo deseen.

\bigskip
Además, este mismo trabajo está publicado bajo la licencia \textbf{Creative Commons Attribution-ShareAlike 4.0}, dando permiso para copiarlo y redistribuirlo en cualquier medio o formato, también de adaptarlo de la forma que se quiera, pero todo esto siempre y cuando se reconozca la autoría y se distribuya con la misma licencia que el trabajo original. El documento en formato {\tt LaTeX} se puede encontrar en el siguiente repositorio de {\tt GitHub}: \url{https://github.com/erseco/ugr_gestion_informacion_web/tree/master/p3/}.

\vspace{4cm}

\noindent Fdo: \autor

\vspace{2cm}

\begin{flushright}
\ciudad, a \today
\end{flushright}

